\section{Simulation}

In order to test the performance of the proposed solution against the current routing solution of the network a payment routing simulator was developed. \\
The simulator is able to work with different routing schemes while working on snapshots of the real lighting network. The simulation has the following free parameters:

\begin{itemize}
    \item Number of payments
    \item Number of nodes
    \item $P \sim \mathcal{N}(\mu,\,\sigma^{2})$
\end{itemize}

The parameter that defines the number of nodes is used to control the size of the network, opening the possibility of making it smaller and thus easier to analyze. Nodes are removed randomly but as a result of working with a scale-free network the network connectivity is kept. The size of the network is reduced without resulting in its fragmentation, this phenomena is studied in \cite{network_science}. \\
$P$, the random variable that represents that amount for each payment follows a normal distribution with free $(\mu,\,\sigma^{2})$ parameters. \\
The distributed routing scheme can be parameterized through:

\begin{itemize}
    \item Number of routing gossip messages
    \item Next hop decision algorithm
    \item Routing table policies
\end{itemize}

It's important to properly set the number of routing message exchanges between each payment, this parameter will define 